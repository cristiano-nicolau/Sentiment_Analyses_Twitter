\section{Conclusion}

This project provided a comprehensive exploration of sentiment analysis on Twitter, allowing for the practical application of a wide range of machine learning and deep learning techniques to a real-world dataset. The central objective was to experiment with multiple model families—including traditional approaches like Support Vector Machines (SVM), deep learning architectures such as Convolutional Neural Networks (CNNs) and Recurrent Neural Networks (RNNs), and state-of-the-art transformer-based models like BERT—while carefully evaluating their performance.

Throughout the process, we critically examined the strengths and limitations of each method, paying particular attention to their behavior when adapted to our specific dataset. The project also highlighted the importance of balancing model complexity with computational constraints, particularly when fine-tuning large-scale pre-trained models.

In the end, our fine-tuned BERT model and the SVM with TF-IDF features achieved the highest performance, both delivering near state-of-the-art accuracy. These results reaffirm the potential of transformer-based models in sentiment analysis tasks, while also demonstrating that, under the right configurations, traditional models can still be highly competitive. 
