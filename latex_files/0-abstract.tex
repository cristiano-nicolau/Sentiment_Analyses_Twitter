\begin{abstract}
This project focuses on Twitter sentiment analysis using the "Twitter Entity Sentiment Analysis" dataset, which contains tweets categorized by their sentiment (e.g., positive, negative, neutral) towards specific entities. We implemented and evaluated a comprehensive suite of models to accurately classify tweet sentiment. These include traditional machine learning approaches such as Support Vector Machines (SVM) with Bag-of-Words (BoW) and TF-IDF features; lexicon-based methods like VADER and TextBlob; various Deep Learning (DL) architectures including Feed-Forward Neural Networks (FNNs), Convolutional Neural Networks (CNNs) for text, Long Short-Term Memory (LSTM) networks, Gated Recurrent Units (GRUs), and Bidirectional LSTMs (Bi-LSTMs); and the transformer-based model BERT. The models were developed with considerations for text pre-processing specific to Twitter data and hyperparameter tuning to optimize performance. This work has implications for applications in public opinion monitoring, brand reputation management, customer feedback analysis, and social media analytics.
\end{abstract}

\begin{center}
    \textit{\textbf{Keywords:} Twitter Sentiment Analysis, Natural Language Processing (NLP), Machine Learning, Deep Learning, Text Classification, SVM, LSTM, CNN, GRU, Bi-LSTM, BERT, VADER, TextBlob}
\end{center}